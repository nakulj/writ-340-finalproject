Located in South Los Angeles, Community Health Councils, Inc. (CHC) works with a wide variety of organizations, people, and communities through many programs to promote better health policies in the Los Angeles community.  We seek to help their organization in its policy recommendations to the City of Los Angeles’ Mobility Element.  Through our research and analysis of complete street practices, we aim to provide insight into ways to improve mobility in the South Los Angeles community.

CHC is producing a report of complete street design guidelines and recommendations to be incorporated into the City of Los Angeles’ Mobility Element.  A key part of this report is detailing the best complete street practices that could be used effectively in Los Angeles.  We identified complete street design principles from around the world, including citations, effectiveness, and images of the best ideas.  We approached the project with several street design principals in mind:  streets are a public space, great streets are great for business, streets need to be designed for the safety of all users, streets can add to green space, and streets should adapt to the current situation.  Finally, we looked at cost to determine the feasibility of implanting these practices.  

	We researched eight different complete street practices used in other parts of the country and world.  These included chokers, curb radius, raised crosswalks, curb extensions, textured pavements, pedestrian and restricted traffic zones, traffic circles, and midblock crossings.  From this, we created a usability index that ranked the practices.  We looked at advantages and disadvantages, effectiveness, and cost and consideration.  The practices with the highest usability rating were curb extensions, curve radii, pedestrian zones, and chokers.  
