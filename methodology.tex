As part of scope of this project, this group has been charged with ranking the previously spotlighted best practices. The purpose of this ranking system is to concretely define which practices, in the opinion of this group, would be most recommended for implementation into the Mobility Element of the city's General Plan as important street design principles.

The Victoria Transport Policy institute lays out a framework for evaluating traffic calming practices. They identify four factors that influence the effectiveness of a particular project:\begin{description}
	\item[Magnitude of Change] How much of an influence a particular measure has on improving pedestrian and cyclist mobility.
	\item[Demand] Improvements to streets are more effective if more people utilize them.  For example, pedestrian facility improvements should be made around busier areas such as schools or commercial centers \cite{TP3}.
	\item[Integration with other improvements]  If only one complete street design practice is applied in an improvement project, it has much less of impact than if many other practices were applied in the same project \cite{TP3}.
	\item[Land use effects] Street improvements that promote pedestrianism can cause changes in land use that further encourage people to walk or cycle, such as shops spring up along busy pedestrian corridors \cite{TP3}.
\end{description}

Since the best practices that have been outlined previously are being examined as general practices, rather than as improvements to specific locations, they will be evaluated based on their individual magnitudes of change, how well they integrate with other practices, and the feasibility of implementing them in the context of Los Angeles. 

To determine each practice's usability index, a score has been assigned to each practice for four characteristics: magnitude of change, integration with other improvements, viability in Los Angeles, and cost. Each practice is rated `low', `medium', or `high'. These scores are derived from the judgment of group members in combination with ratings of a similar nature from San Francisco's Metropolitan Transportation Commission \cite{PZ7}. The Usability Index score is an agglomeration of these four scores, also with ratings of low', `medium', or `high'.