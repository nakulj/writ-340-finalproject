A city's choice of guiding principles behind street design has a large impact on the daily life of its people. This is especially true in an area like Los Angeles, where so many depend on the ubiquitous roads to get between home, work, and school. Thus, it is important that decisions with regard to policy must be made keeping in mind the needs of all parties involved: commuters, pedestrians, residents, and local business owners.

The research that we have done into the `best practices' currently in use at various locations around the world has revealed telling data on the effects these measures have had on their neighborhoods. The existence of these quantified metrics allowed us to evaluate these practices in an objective manner, and determine which practices would be appropriate for the Los Angeles area.

We hope that our findings will assist the CHC with selecting and promoting better mobility policies. Should these be succesfully petitioned to our policymakers, they will translate into streets that are safer and cleaner.