Upon reflection, select individual ratings stand out, specifically, the `low' rating of pedestrian zones for viability in Los Angeles. The group felt that one particular challenge with implementing pedestrian zones was that in most examples, they were typically put in place in areas with high population density. In the context of Los Angeles, and specifically South Los Angeles, the group felt that it would be difficult to select a location for a pedestrian zone, and that pedestrian traffic should instead first be built up in such a manner that a pedestrian zone would be more accessible when it is finally implemented. That said, the group ultimately rated pedestrian zones with `high' usability index due to the significant benefits they can provide to public health.

Research into the subject found there is no consensus on the ability of textured or colored pavements to reduce vehicle speeds, which significantly contributed to its `low' usability index rating; however, most examples of textured pavement found by the group were in combination with other street elements, such as raised crosswalks, traffic circles, or pedestrian zones. The visual cue that textured and colored pavements provide to drivers and pedestrians, combined with potential aesthetic benefits, make them good accents for other projects, but not viable as a standalone improvement. 

In addition to the previously discussed pedestrian zones, the group finds chokers, curve radii, and curb extensions to be of high usability to the city. The commonality between these practices is that they contribute to mobility by creating safer environments for pedestrians, and are extremely viable in a city such as Los Angeles, where there are numerous available location for these practices to be implemented.